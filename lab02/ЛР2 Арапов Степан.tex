\documentclass{article}
\usepackage[utf8]{inputenc}
\usepackage[russian]{babel}

\usepackage{sagetex}
\setlength{\sagetexindent}{10ex}

\usepackage[left=15mm,top=30mm,bottom=30mm,right=15mm]{geometry}

\begin{document}

\begin{center}
\Large{\textbf{Арапов Степан , М8О-208б-19}}

\Large{\textbf{Лабораторная работа 2. Третья часть.}}
\end{center}

\textbf{Задание:}
\begin{enumerate}
\item Привести поверхность, заданную уравнением, к каноническому виду.
\item Построить исходную поверхность и поверхность в каноническом виде.
\item Собственные числа и вектора рассчитать вручную, сравнить с результатом встроенных функций.
\end{enumerate}

По списку 2, Вариант 2: 
$f(x, y, z) = 6x^2 + 12xy + 7y^2 + 2xz + 3z^2 + 5x + 5y + 5z - 18$
\newline

\textbf{Построение исходной поверхности.}
\begin{sageblock}
f(x, y, z) = 6*x^2 + 12*x*y + 7*y^2 + 2*x*z + 3*z^2 + 5*x + 5*y + 5*z - 18
\end{sageblock}

Вот так теперь выглядит наша функция:
$\sage{f}$

Построим исходную поверхность.

\begin{center}
\sageplot{implicit_plot3d(f(x=x, y=y, z=z), (x, -10, 10), (y, -10, 10), (z, -10, 10),  figsize=3)}
\end{center}
\textbf{Приведение поверхности к каноническому виду.} \newline
Создадим две матрицы: матрица $A$ для квадратичной формы, матрица $B$ из коэффициентов квадратичной и линейной формы и из свободного члена.
\begin{sageblock}
A = matrix([
    [6, 6, 1],
    [6, 7, 0],
    [1, 0, 3]
])

B = matrix([
    [6, 6, 1, 2.5],
    [6, 7, 0, 2.5],
    [1, 0, 3, 2.5],
    [2.5, 2.5, 2.5, -18]
])
\end{sageblock}
Посчитаем ортогональные инварианты.
\begin{sageblock}
t1 = A.trace()
t2 = A[0:2, 0:2].det() + A[[0, 2], [0, 2]].det() + A[1:3, 1:3].det()
d = A.det()
delta = B.det()
\end{sageblock}
Результаты вычислений: $t1 = \sage{t1},   t2 = \sage{t2},   d = \sage{d},  delta = \sage{delta}$ \newline
Найдём собственные значения матрицы А.

\begin{sageblock}
E = matrix.identity(3)
eigenvalues = []

var("lmbda")
for eigenvalue in solve((A - lmbda * E).det() == 0, lmbda):
    show(n(eigenvalue.rhs()))
    eigenvalues.append(eigenvalue.rhs())
\end{sageblock}
Результат в символьном и численном виде:
$$\sage{eigenvalues[0]} == \sage{eigenvalues[0].n()}$$
$$\sage{eigenvalues[1]} == \sage{eigenvalues[1].n()}$$
$$\sage{eigenvalues[2]} == \sage{eigenvalues[2].n()}$$
 Или другой способ:
\begin{sageblock}
A.eigenvalues()
\end{sageblock}
$$\sage{A.eigenvalues()[0]}, \sage{A.eigenvalues()[1]}, \sage{A.eigenvalues()[2]}$$\newline
Получили одинаковые значения (c незначительной погрешностью), значит посчитано верно. \newline
Перейдём к последнему шагу, а именно к построениею поверхности в каноническом виде.
Для начала нужно составить каноническое уравнение.
\begin{sageblock}
f_canonical(x,y,z) = x**2 * A.eigenvalues()[0] +
  y**2 * A.eigenvalues()[1] + z**2 * A.eigenvalues()[2]  + delta / d
\end{sageblock}
Получили уравнение:
$$\sage{f_canonical}$$
Наконец изобразим:
\begin{center}
\sageplot{implicit_plot3d(f_canonical, (x, -10, 10), (y, -10, 10), (z, -10, 10), figsize=3)}
\end{center}

\end{document}